\section{Título:}
Fenomenología de modelos escotogénicos
\section{Investigador principal y coinvestigadores: }
UdeA: Diego Alejandro Restrepo Quintero,
Oscar Alberto Zapata Noreña, William Ponce, y Daniel Jaramillo. 
ITM: Richard Benavides, Luis A. Muñoz. 
UniAndes: Carlos Ávila y  Juan Carlos Sanabria.

\section{Conformación del equipo investigador }
\begin{colciencias}
Colocar el nombre y código, registrado en el GrupLac, del o de los grupos de investigación. Al igual que el nombre de los demás integrantes que conforman el equipo de trabajo. Se debe incluir el tiempo de dedicación y funciones en el marco del proyecto.
\end{colciencias}
Grupo de Fenomenología de las Interacciones Fundamentales COL0008423.
...

Postdoc: Andres Florez (UniAndes)


Asesores
\begin{itemize}
\item Raffaele Fazio (UNAL)
\item Carlos Yaguna (Munster)
\item Diego Aristizabal (Liege)
\item Enrico Nardi (Frascati)
\item José Valle (Valencia)
\item Martin Hirsch (Valencia)
\end{itemize}


\section{Antecedentes y resultados previos del equipo de investigación solicitante en la temática específica del proyecto}
\begin{colciencias}
Trayectoria del equipo de investigación con relación al problema planteado en el proyecto.
\end{colciencias}
El grupo de la UdeA ha participado activamente en esta línea con
investigaciones sobre la correlación entre física de neutrinos y
observables en aceleradores (intersección entre fronteras de energía e
intensidad)~\cite{Restrepo:2013aga}, correciones importantes a la sección eficaz de
detección directa de materia oscura escalar (frontera
cósmica)~\cite{Klasen:2013btp}, coaniquilaciones entre neutrinos derechos y
materia oscura escalar (intersección entre fronteras de intensidad y
cósmica)~\cite{Klasen:2013jpa} y en la generalización de los modelos escotogénicos
(intersección entre las tres fronteras)~\cite{Restrepo:2013aga}. Los
trabajos del grupo al respecto han acumulado más de 70 citaciones, la
mitad de ellas en el último año: \url{http://goo.gl/YRAJmJ}.

...

\section{Temática de investigación}
\begin{evaluacioncodi}
  Agenda de investigación: (este ítem no tiene puntaje)
¿Considera que el proyecto evaluado está claramente vinculado y aporta a una agenda o línea de
investigación?
\end{evaluacioncodi}
\begin{colciencias}
 Especificar en cuál de las temáticas definidas por la convocatoria está enmarcado el proyecto.
\end{colciencias}
Investigación fundamental. 
\section{Resumen ejecutivo}
\begin{colciencias}
Información mínima necesaria para comunicar de manera precisa los contenidos y alcances del proyecto.
\end{colciencias}
Recientemente se ha encontrado todo el conjunto de modelos
escotogénicos los cuales contienen un candidato de materia oscura que
a su vez participa en la generacion de masas de neutrinos a un
loop. Con el desarrollo de este proyecto se busca tener un marco más
general para la búsqueda de señales de producción electrodébil con
alta energía faltante en el LHC que amplien las posibilidades de
descubrir señales de nueva física.
\section{Palabras clave:}
\begin{colciencias}
Incluir máximo seis (6) palabras clave que describan el objeto del proyecto. 
\end{colciencias} 
Física de partículas, masas de neutrinos, materia oscura, más allá del modelo estándar.